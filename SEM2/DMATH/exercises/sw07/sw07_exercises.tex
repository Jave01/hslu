\documentclass[12pt]{scrartcl}
\usepackage[ngerman]{babel}


\usepackage{amsmath, amssymb}

\usepackage{array}  % for the tables

\usepackage{nameref}  % for referencing with name

\usepackage{hyperref}  % for hyperlinks

\usepackage{mathrsfs}

\usepackage{graphicx}  % for the images

\usepackage{xcolor, colortbl}

\usepackage{gensymb} % for \degree

\usepackage{pgfplots}

\usepackage{tabto}

\usepackage{ulem} % \uuline

\usetikzlibrary{arrows}

% \usepgfplotslibrary{external}

% \tikzexternalize

\definecolor{Gray}{gray}{0.85}

\setlength{\parindent}{0cm}

\newcommand{\RomanNumeralCaps}[1]
    {\MakeUppercase{\romannumeral #1}}

% hyperlinks
\hypersetup{
    colorlinks,
    citecolor=black,
    filecolor=black,
    linkcolor=black,
    urlcolor=black
}

\bibliographystyle{IEEetran}




\author{David Jäggli}

\title{Diskrete Mathematik - Übungen SW07}



% ---------- Begin Main Document ----------- %



\begin{document}

\maketitle

\tableofcontents

\newpage
\section{Binomialverteilung und hypergeometrische Verteilung}
\textbf{I.)}\\
a) $B(10|1000,0.01) = \binom{1000}{10}0.01^{10} \cdot 0.99^{990} = 12.57\%$\\
b) $\sum_{k=0}^{9}B(k|1000,0.01) = \sum_{k=0}^{9}\binom{1000}{k}0.01^k \cdot 0.99^{1000-k} = 45.8\%$\\
b) $\sum_{k=0}^{20} 1 - B(k|1000,0.01) = 1 - \sum_{k=0}^{20} \binom{1000}{k}0.01^k \cdot 0.99^{1000-k} = 0.15\%$\\


\section{Poissonverteilung}
\textbf{II.)}\\
$P(k) = \frac{e^{-\mu} \cdot \mu^k}{k!}$, $\mu = 3$\\

a) $P(k=0) = \frac{3^0}{0!} \cdot e^{-3} = 5\%$\\
b) $P(k=1) = \frac{3^1}{1!} \cdot e^{-3} = 14.9\%$\\
c) $P(k \leq 2) = 1 - \sum_{k=0}^{2} \frac{3^k \cdot e^{-3}}{k!} = 42.3\%$\\
d) $P(k > 2) = 1 - \sum_{k=0}^{2} \frac{3^k \cdot e^{-3}}{k!} = 57.7\%$

\section{Zufallsvariablen, Erwartungswerte und Varianzen}
\textbf{III.)}\\
a) Wahrscheinlichkeit für schwarz: $\frac{2}{5}$\\


\vspace{1.5cm}
\textbf{IV.)}\\
Beispielverteilung für Veranschaulichung bei n=3:
\renewcommand{\arraystretch}{1.5}
\begin{center}
    \begin{tabular}{  c | c | c  }
        & $Y$ & \\ 
        $X$ & 0 1 2 3 & $P(X)$\\ 
        \hline
        0 & 0 0 0 $\frac{1}{8}$ & $\frac{1}{8}$\\ 
        1 & 0 0 $\frac{3}{8}$ 0 & $\frac{3}{8}$\\ 
        2 & 0 $\frac{3}{8}$ 0 0 & $\frac{3}{8}$\\ 
        3 & $\frac{1}{8}$ 0 0 0 & $\frac{1}{8}$\\ 
        \hline
        $P(Y)$ & $\frac{1}{8}$ $\frac{3}{8}$ $\frac{3}{8}$ $\frac{1}{8}$ & \\ 
    \end{tabular}
\end{center}

$P(X = 0 \land Y = 3) = \frac{1}{8} \neq \frac{1}{8} \cdot \frac{1}{8}$

\newpage
\textbf{V.)}\\
Falls $X = $ Würfe bis Augenzahl = 7, dann gilt folgendes:\\
\[P(X=k) = \left(1-\frac{1}{6}\right)^{k-1} \cdot \frac{1}{6} = \left(\frac{5}{6}\right)^{k-1} \frac{1}{6}\]


Die durchschnittliche Anzahl der Würfe bis eine 7 gewürfelt wird ist:

\[E(X) = \sum_{k=0}^{\infty} k \left(\frac{5}{6}\right)^{k-1} \frac{1}{6} = \frac{1}{6} \sum_{k=0}^{\infty} k \left( \frac{5}{6} \right)^{k-1} = \frac{1}{6} \frac{1}{(1-\frac{5}{6})^2} = 6\]



% Matrix example
% \textbf{Korrektur:}\\
% $\mathbf{A} \odot \mathbf{B} =
% \begin{bmatrix}
%     1 & 1 & 1 \\
%     1 & 1 & 1 \\
%     1 & 0 & 1 \\
% \end{bmatrix}
% $

% tabular example 3 columns
% \renewcommand{\arraystretch}{1.5}
% \begin{center}
%     \begin{tabular}{ | m{12em} | m{12em} | m{12em} | }
%         \hline
%         1 & 2 & 3\\ 
%         \hline
%         1 & 2 & 3\\ 
%         \hline
%         1 & 2 & 3\\ 
%         \hline
%     \end{tabular}
% \end{center}


% tabular example 2 columns
% \renewcommand{\arraystretch}{1.5}
% \begin{center}
%     \begin{tabular}{ | m{17em} | m{17em} | }
%         \hline
%         1 & 2\\ 
%         \hline
%         1 & 2\\ 
%         \hline
%         1 & 2\\ 
%         \hline
%     \end{tabular}
% \end{center}

% \bibliography{}

\end{document}