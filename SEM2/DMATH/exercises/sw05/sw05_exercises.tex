\documentclass[12pt]{scrartcl}
\usepackage[ngerman]{babel}


\usepackage{amsmath, amssymb}

\usepackage{array}  % for the tables

\usepackage{nameref}  % for referencing with name

\usepackage{hyperref}  % for hyperlinks

\usepackage{mathrsfs}

\usepackage{graphicx}  % for the images

\usepackage{xcolor, colortbl}

\usepackage{gensymb} % for \degree

\usepackage{pgfplots}

\usepackage{tabto}

\usepackage{ulem} % \uuline

\usetikzlibrary{arrows}

% \usepgfplotslibrary{external}

% \tikzexternalize

\definecolor{Gray}{gray}{0.85}

\setlength{\parindent}{0cm}

\newcommand{\RomanNumeralCaps}[1]
    {\MakeUppercase{\romannumeral #1}}

% hyperlinks
\hypersetup{
    colorlinks,
    citecolor=black,
    filecolor=black,
    linkcolor=black,
    urlcolor=black
}

\bibliographystyle{IEEetran}




\author{David Jäggli}

\title{Diskrete Mathematik - Übungen SW05}



% ---------- Begin Main Document ----------- %



\begin{document}

\maketitle

\tableofcontents

\newpage
\section{Grundlagen des Zählens}
\textbf{I.)}\\

$26^4 + 1 = 456'976$\\

\textbf{Falsch, da Wörter auch 1,2 oder 3 lang sein können}\\

$\displaystyle{\sum_{k=0}^{4} 26^j = 475'255}$\\
\vspace{20px}

\textbf{II.)}\\
$2^{10-3} + 2^{10-2} = 384$\\

Kann es sein, dass die Lösung hier falsch ist? Weil in der Aufgabenstellung wird von 
\textit{oder} gesprochen, in der Lösung jedoch werden die beiden Mengen \textit{und} verknüpft.\\


\section{Schubfachprinzip}
\textbf{III.)}\\
Bei einer Division durch 4 ergeben sich Reste von 0, 1, 2 oder 3. Man hat jedoch
5 Zahlen, welche sich auf diese 4 Zahlen aufteilen müssen $\rightarrow$ es gibt mindestens
zwei mal den gleichen Rest.


\textbf{IV.)}\\
Maximal 7. \\
Worst case: Man nimmt immer abwechslungsweise einen roten dann einen blauen Socken.

ceil(7/2) = 4\\


\section{Permutationen und Kombinationen}
\textbf{V.)}\\
?\\



\textbf{VI.)}\\
Immer 2 Null nach 1 $\rightarrow$ Objekte: 0b100\\
Anzahl 1: 4\\
Anzahl 0: 12\\

$4 \cdot 2$ = 8 $\rightarrow$ 4 verschiebbare 0s\\

Objekt: O\\

Mögliche Plätze für Nullen: \text{O\_\_\_\_O\_\_\_\_O\_\_\_\_O\_\_\_\_}\\

Permutation ohne Wiederholung:
$\frac{4!}{(4-4)!} = 4*3*2*1 = 24$

\textbf{Ke ahnig wasi fausch gmacht ha}\\


\textbf{VII.)}\\
Kann nur 49 geben, wenn 2x 7 und anderenfalls Einsen vorkommen.
Heisst für $10^1$ Ziffern: 1\\
für $10^2$ Ziffern: 3\\
für $10^3$ Ziffern: 6\\
für $10^4$ Ziffern: 10\\
für $10^5$ Ziffern: 15\\

ergibt 35


\textbf{VIII.)}\\
Geht nur wenn jede Zahl stimmt, nur Reihenfolge ist egal.


12 = 1+11\\
12 = 2+10\\
12 = 3+9\\
12 = 4+8\\
12 = 5+7\\
12 = 6+6\\
12 = 1+2+9\\
12 = 1+3+8\\
12 = 1+4+7\\
12 = 1+5+6\\
12 = 1+5+6\\
\dots\\
12 = 1 + 1 + 1 + 1 + 1 + 1 + 1 + 1 + 1 + 1 + 1 + 1\\
\dots\\

Wenn man jede Ziffer in Zahlen $< 10^6$ als mögliche Stelle ansieht, welche einen
Wert von 0-9 annehmen kann, ergibt sich folgender Binomialkoeffizient:\\
$\binom{12+5}{5} = 6188$\\

\textbf{Korrektur:} Randbedingungen\dots 6062

\newpage
\textbf{IX.)}\\
W: 2\\
E: 4\\
T: 3\\
B: 2\\
R: 2\\
S: 4\\

$n = 17$\\

$\displaystyle{\frac{17!}{4!4!3!2!2!2!}}$

% Matrix example
% \textbf{Korrektur:}\\
% $\mathbf{A} \odot \mathbf{B} =
% \begin{bmatrix}
%     1 & 1 & 1 \\
%     1 & 1 & 1 \\
%     1 & 0 & 1 \\
% \end{bmatrix}
% $

% tabular example 3 columns
% \renewcommand{\arraystretch}{1.5}
% \begin{center}
%     \begin{tabular}{ | m{12em} | m{12em} | m{12em} | }
%         \hline
%         1 & 2 & 3\\ 
%         \hline
%         1 & 2 & 3\\ 
%         \hline
%         1 & 2 & 3\\ 
%         \hline
%     \end{tabular}
% \end{center}


% tabular example 2 columns
% \renewcommand{\arraystretch}{1.5}
% \begin{center}
%     \begin{tabular}{ | m{17em} | m{17em} | }
%         \hline
%         1 & 2\\ 
%         \hline
%         1 & 2\\ 
%         \hline
%         1 & 2\\ 
%         \hline
%     \end{tabular}
% \end{center}

% \bibliography{}

\end{document}