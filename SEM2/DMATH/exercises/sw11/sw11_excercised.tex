\documentclass[12pt]{scrartcl}
\usepackage[ngerman]{babel}


\usepackage{amsmath, amssymb}

\usepackage{array}  % for the tables

\usepackage{nameref}  % for referencing with name

\usepackage{hyperref}  % for hyperlinks

\usepackage{mathrsfs}

\usepackage{graphicx}  % for the images

\usepackage{xcolor, colortbl}

\usepackage{gensymb} % for \degree

\usepackage{pgfplots}

\usepackage{tabto}

\usepackage{ulem} % \uuline

\usetikzlibrary{arrows}

% \usepgfplotslibrary{external}

% \tikzexternalize

\definecolor{Gray}{gray}{0.85}

\setlength{\parindent}{0cm}

\newcommand{\RomanNumeralCaps}[1]
    {\MakeUppercase{\romannumeral #1}}

% hyperlinks
\hypersetup{
    colorlinks,
    citecolor=black,
    filecolor=black,
    linkcolor=black,
    urlcolor=black
}

\bibliographystyle{IEEetran}




\author{David Jäggli}

\title{Diskrete Mathematik - Übungen SW11}



% ---------- Begin Main Document ----------- %


\begin{document}

\maketitle

\tableofcontents

\newpage
\section{Einführung in die Zahlentheorie III}
\textbf{I.)}\\
Weis nicht ob das genügt, aber ist der output 00, ist automatisch der input 1 $\rightarrow$ 
keine Sicherheit in diesem Fall.


\vspace{1.5cm}
\textbf{II.)}\\
a)\\
$\phi (pq) = (p-1)(q-1)$\\
$\phi (47 \cdot 59) = (47-1)(59-1) = 46 \cdot 58 = 2668$\\

$2668 = 156 \cdot 17 + 16$\\
$17 = 1 \cdot 16 + 1$\\

Sie sind teilerfremd.\\

b)\\
Modulares Inverses von $e$ mod $\phi(pq) = 17 \mod 2668$\\
$d \cdot e \mod \phi(pq) = 1$\\
$d \cdot e + x \cdot \phi(pq) = 1$\\

$1 = 17 - 1 \cdot 16$\\
$1 = 17 - (2668 - 156 \cdot 17)$\\
$1 = 157 \cdot 17 + (-1) \cdot 2668$\\
$d = 157$\\

\begin{align*}
    e\ d            &\equiv 1 \mod \phi(pq)\\
    17 \cdot 157    &\equiv 1 \mod 2668\\
    2669            &\equiv 1 \mod 2668\\
\end{align*}

true


\newpage
c)\\
$e: 17$\\
$n: 2773$\\
$m_1: 8$ $m_2: 117$ $m_3: 1212$\\
$c \equiv m^e \mod n$
\begin{align*}
    m_1 \rightarrow c_1 &= 8^{17} \mod 2773 = 596\\
    m_2 \rightarrow c_2 &= 117^{17} \mod 2773 = 1769\\
    m_3 \rightarrow c_3 &= 1212^{17} \mod 2773 = 2345\\
\end{align*}

\vspace{0.5cm}
d)\\
$d = 157$\\
$n = 2773$
\begin{align*}
    c_1 \rightarrow m_1 &= 596^{157} \mod 2773 = 8\\
    c_2 \rightarrow m_2 &= 1769^{157} \mod 2773 = 117\\
    c_3 \rightarrow m_3 &= 2345^{157} \mod 2773 = 1212\\
\end{align*}


\textbf{III.)}\\
$n = 17'753$\\
$\phi (n) = 17280$\\

$\phi (n) = (p-1)(q-1)$\\
$n = p \cdot q \rightarrow q = \frac{n}{p}$\\

2 Unbekannte, 2 Gleichungen\\
\begin{align*}
    (p-1)(q-1)                                  &= \phi (n)\\
    (p-1)\left(\frac{n}{p}-1\right)             &= \phi (n)\\
    n - p - \frac{n}{p} + 1                     &= \phi (n)\\
    n - p - \frac{n}{p} + 1 - \phi (n)          &= 0\\
    -p^2 + np - n + p - \phi (n) \cdot p        &= 0\\
    -p^2 + (n - \phi (n) + 1)p - n              &= 0\\
    -p^2 + 474p - 17'753                        &= 0\\
\end{align*}

$p_1 = 41$\\
$p_2 = 433$\\


\vspace{0.5cm}
\textbf{IV.)}\\
Schlüssel: $(n, e) = (2537, 13)$\\
Geheimtext: $c = 2018$\\

$n$ faktorisieren $\rightarrow n = 43 \cdot 59$\\
$\phi (n) = (43-1)(59-1) = 42 \cdot 58 = 2436$\\

Modular Inverse $d$ von $e$ mod $\phi (n)$\\
Mi erweitertertem euklidischem Algorithmus $\rightarrow d \cdot e + x \cdot \phi (n) = 1$\\

$1 = -5 \cdot 2436 + 937 \cdot 13$\\

Mit modulo rechnen und es folgt:
$d \cdot e = 937 \cdot 13 \equiv 1 \mod 2436$\\
$d = 937$\\

Für Klartext berechnen, Formel: $M = C^d \mod n$ anwenden.\\
$M = C^d \mod n = 2081^{937} \mod 2537 = 1819$



% \begin{align*}
%     1 &= 1 \\
% \end{align*}


% Abschnittsweise definierte Funktionen
% \[ y = g(x) = 
%     \begin{cases} 
%     \frac{1}{2}x    & x \in ]-\infty; -2] \\
%     -2x+3           & x \in ]-2; 3]\\
%     5               & x \in ]3;\infty[
%  \end{cases}
% \]

% Matrix example
% \textbf{Korrektur:}\\
% $\mathbf{A} \odot \mathbf{B} =
% \begin{bmatrix}
%     1 & 1 & 1 \\
%     1 & 1 & 1 \\
%     1 & 0 & 1 \\
% \end{bmatrix}
% $

% tabular example 3 columns
% \renewcommand{\arraystretch}{1.5}
% \begin{center}
%     \begin{tabular}{ | m{12em} | m{12em} | m{12em} | }
%         \hline
%         1 & 2 & 3\\ 
%         \hline
%         1 & 2 & 3\\ 
%         \hline
%         1 & 2 & 3\\ 
%         \hline
%     \end{tabular}
% \end{center}


% tabular example 2 columns
% \renewcommand{\arraystretch}{1.5}
% \begin{center}
%     \begin{tabular}{ | m{17em} | m{17em} | }
%         \hline
%         1 & 2\\ 
%         \hline
%         1 & 2\\ 
%         \hline
%         1 & 2\\ 
%         \hline
%     \end{tabular}
% \end{center}

% \bibliography{}

\end{document}