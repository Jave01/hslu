\documentclass[12pt]{scrartcl}
\usepackage[ngerman]{babel}


\usepackage{amsmath, amssymb}

\usepackage{array}  % for the tables

\usepackage{nameref}  % for referencing with name

\usepackage{hyperref}  % for hyperlinks

\usepackage{mathrsfs}

\usepackage{graphicx}  % for the images

\usepackage{xcolor, colortbl}

\usepackage{gensymb} % for \degree

\usepackage{pgfplots}

\usepackage{tabto}

\usepackage{ulem} % \uuline

\usetikzlibrary{arrows}

% \usepgfplotslibrary{external}

% \tikzexternalize

\definecolor{Gray}{gray}{0.85}

\setlength{\parindent}{0cm}

\newcommand{\RomanNumeralCaps}[1]
    {\MakeUppercase{\romannumeral #1}}

% hyperlinks
\hypersetup{
    colorlinks,
    citecolor=black,
    filecolor=black,
    linkcolor=black,
    urlcolor=black
}

\bibliographystyle{IEEetran}




\author{David Jäggli}

\title{Diskrete Mathematik - Übungen SW08}



% ---------- Begin Main Document ----------- %



\begin{document}

\maketitle

\tableofcontents

\newpage
\section{Rekursionen}
\textbf{I.)}\\
Gleichung: $a_n = 2^n + 5 \cdot 3^n$\\
RB: $a_n = 5a_{n-1} - 6a_{n-2}$\\

Einsetzen:

$5(2^{n-1} + 5 \cdot 3^{n-1}) - 6(2^{n-2} + 5 \cdot 3^{n-2}) = $\\
$5 \cdot 2^{n-1} + 25 \cdot 3^{n-1} - 6 \cdot 2^{n-2} - 30 \cdot 3^{n-2} = $\\
$2^{n-2} (5 \cdot 2 - 6) + 3^{n-2} (25 \cdot 3 - 30) = $\\
$2^{n-2} \cdot 4 + 3^{n-2} \cdot 45 = $\\
$2^{n-2} \cdot 2^2 + 3^{n-2} \cdot 5 \cdot 3^2 = $\\
$2^n + 5 \cdot 3^n = a_n$, $\forall n$\\


\textbf{II.)}\\
Wenn mit 1 endet dann gibt es $n^{-1} Möglichkeiten$


% Matrix example
% \textbf{Korrektur:}\\
% $\mathbf{A} \odot \mathbf{B} =
% \begin{bmatrix}
%     1 & 1 & 1 \\
%     1 & 1 & 1 \\
%     1 & 0 & 1 \\
% \end{bmatrix}
% $

% tabular example 3 columns
% \renewcommand{\arraystretch}{1.5}
% \begin{center}
%     \begin{tabular}{ | m{12em} | m{12em} | m{12em} | }
%         \hline
%         1 & 2 & 3\\ 
%         \hline
%         1 & 2 & 3\\ 
%         \hline
%         1 & 2 & 3\\ 
%         \hline
%     \end{tabular}
% \end{center}


% tabular example 2 columns
% \renewcommand{\arraystretch}{1.5}
% \begin{center}
%     \begin{tabular}{ | m{17em} | m{17em} | }
%         \hline
%         1 & 2\\ 
%         \hline
%         1 & 2\\ 
%         \hline
%         1 & 2\\ 
%         \hline
%     \end{tabular}
% \end{center}

% \bibliography{}

\end{document}