\documentclass[12pt]{scrartcl}
\usepackage[ngerman]{babel}


\usepackage{amsmath, amssymb}

\usepackage{array}  % for the tables

\usepackage{nameref}  % for referencing with name

\usepackage{hyperref}  % for hyperlinks

\usepackage{mathrsfs}

\usepackage{graphicx}  % for the images

\usepackage{xcolor, colortbl}

\usepackage{gensymb} % for \degree

\usepackage{pgfplots}

\usepackage{tabto}

\usepackage{ulem} % \uuline

\usetikzlibrary{arrows}

% \usepgfplotslibrary{external}

% \tikzexternalize

\definecolor{Gray}{gray}{0.85}

\setlength{\parindent}{0cm}

\newcommand{\RomanNumeralCaps}[1]
    {\MakeUppercase{\romannumeral #1}}

% hyperlinks
\hypersetup{
    colorlinks,
    citecolor=black,
    filecolor=black,
    linkcolor=black,
    urlcolor=black
}

\bibliographystyle{IEEetran}




\author{David Jäggli}

\title{Diskrete Mathematik - Übungen SW01}



% ---------- Begin Main Document ----------- %



\begin{document}

\maketitle

\tableofcontents

\newpage
\section{Funktionen}
\textbf{I.)}\\
$\mathbb{D} = \mathbb{R}$\\
$\mathbb{W} = \mathbb{N}$\\

\textbf{Korrektur I.)}\\
$\mathbb{D} =$ alle binäre Strings\\
$\mathbb{W} = 2\mathbb{N}$\\


\textbf{II.)}\\
Gegeben:\\
$f(x) = x^2 + 1$\\
$g(x) = x + 2$\\
whereas $\mathbb{R} \mapsto \mathbb{R}$\\

Gesucht:
\begin{enumerate}
    \item $f \circ g$
    \item $g \circ f$
    \item $f + g$
    \item $f \cdot g$\\
\end{enumerate}

$f \circ g = f(x + 2) = (x + 2)^2 + 1 =$ \underline{$x^2 + 4x + 5$}\\
$g \circ f = g(x^2 + 1) = (x^2 + 1) + 2 =$ \underline{$x^2 + 3$}\\
$f + g = (x^2 + 1) + (x + 2) =$ \underline{$x^2 + x + 3$}\\
$f \cdot g = (x^2 + 1) \cdot (x + 2) =$ \underline{$x^3 + 2x^2 + x + 2$}\\


\textbf{III.)}\\
a)\\
$\displaystyle{\prod_{i=0}^{10} i} = 0$\\

b)\\

$\displaystyle{\prod_{i=5}^{8} i} = 1680$\\

c)\\

$\displaystyle{\prod_{i=1}^{100} (-1)^i} = -1$\\

d)\\

$\displaystyle{\prod_{i=1}^{100} (-1)^i} = -1$\\

\textbf{Korrektur:}\\
c)
% tabular example 3 columns
% \renewcommand{\arraystretch}{1.5}
% \begin{center}
%     \begin{tabular}{ | m{12em} | m{12em} | m{12em} | }
%         \hline
%         1 & 2 & 3\\ 
%         \hline
%         1 & 2 & 3\\ 
%         \hline
%         1 & 2 & 3\\ 
%         \hline
%     \end{tabular}
% \end{center}


% tabular example 2 columns
% \renewcommand{\arraystretch}{1.5}
% \begin{center}
%     \begin{tabular}{ | m{17em} | m{17em} | }
%         \hline
%         1 & 2\\ 
%         \hline
%         1 & 2\\ 
%         \hline
%         1 & 2\\ 
%         \hline
%     \end{tabular}
% \end{center}

% \bibliography{}

\end{document}