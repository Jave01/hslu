\documentclass[12pt]{scrartcl}
\usepackage[ngerman]{babel}


\usepackage{amsmath, amssymb}

\usepackage{array}  % for the tables

\usepackage{nameref}  % for referencing with name

\usepackage{hyperref}  % for hyperlinks

\usepackage{mathrsfs}

\usepackage{graphicx}  % for the images

\usepackage{xcolor, colortbl}

\usepackage{gensymb} % for \degree

\usepackage{pgfplots}

\usepackage{tabto}

\usepackage{ulem} % \uuline

\usetikzlibrary{arrows}

% \usepgfplotslibrary{external}

% \tikzexternalize

\definecolor{Gray}{gray}{0.85}

\setlength{\parindent}{0cm}

\newcommand{\RomanNumeralCaps}[1]
    {\MakeUppercase{\romannumeral #1}}

% hyperlinks
\hypersetup{
    colorlinks,
    citecolor=black,
    filecolor=black,
    linkcolor=black,
    urlcolor=black
}

\bibliographystyle{IEEetran}




\author{David Jäggli}

\title{Diskrete Mathematik - Übungen SW06}



% ---------- Begin Main Document ----------- %



\begin{document}

\maketitle

\tableofcontents

\newpage
\section{Wahrscheinlichkeitstheorie}
\textbf{I.)}\\
Kombinationen mit 2 Würfeln: \{(3,6), (4,5), (5,4), (6,3)\}\\
Möglichkeiten mit 2 Würfeln: $6^2$\\
Wahrscheinlichkeit mit 2 Würfeln: $\frac{4}{36} = \frac{1}{9} \approx 11.11\%$\\

Kombinationen mit 3 Würfeln: \{(1,2,6), (1,3,5), (1,4,4), (2,3,4) \dots\}: 25 Kombinationen\\
Möglichkeiten mit 3 Würfeln: $6^3$\\
Wahrscheinlichkeit mit 3 Würfeln: $\frac{25}{216} \approx 11.57\%$\\

Antwort: Wahrscheinlichkeit ist höher mit 3 Würfeln.\\
\vspace{25px}

\textbf{II.)}\\
Möglichkeiten: $\displaystyle{\binom{90}{10}}$\\

3 verschiedene Fälle wie Resultat erzielt werden kann: 
\begin{itemize}
    \item Keine rote Kugel
    \item Keine weisse Kugel
    \item Keine blaue Kugel
\end{itemize}

Fall 1 (keine rote Kugel) dann müssen die 10 Plätze unter weiss und blau ausgehandelt werden.\\
Für eine der beiden Farben (z.B. weiss) ist die Wahrscheinlichkeit:

\[\binom{30 + 30 + 30}{n}\]

Das bedeutet, dass blau die anderen Plätze ausfüllen muss was in folgendem resultiert:

\[\binom{30 + 30 +30}{10 - n}\]

Wobei n von 0 bis 10 geht.\\

Die Anzahl valide Möglichkeiten ergibt sich aus der Summe der beiden Fälle:

\[\sum_{n=0}^{10} \binom{90}{n} \binom{90}{10 - n}\]
\newpage

Was in folgender Wahrscheinlichkeit für Rot resultiert:
\[\frac{\sum_{n=0}^{10} \binom{90}{n} \binom{90}{10 - n}}{\binom{90}{10}}\]

Vandermon: \[\frac{\binom{60}{10}}{\binom{90}{10}} = 0.0132\]

Für den zweiten und dritten Fall gilt genau das Gleiche, heisst die komplette
Wahrscheinlichkeit ergibt sich aus der Summe der drei Fälle:

$p(A_r) \cup p(A_w) \cup p(A_b) = p(A_r) + p(A_w) + p(A_b) - p(A_r \cap A_w) - p(A_r \cap A_b)$ \\
$- p(A_w \cap A_b) + p(A_r \cap A_w \cap A_b) = 0.039$\\
\vspace{20px}

\section{Bedingte Wahrscheinlichkeit und Unabhängigkeit}
\textbf{III.)}\\
a) Anzahl Kombinationen 3 gleiche: $\binom{5}{3} = 10$\\

Wahrscheinlichkeit für 3 Jungen = $0.51^3 \cdot 0.49^2 = 0.0319 \Rightarrow 10 * 0.0319 = 31.9\%$\\

b) $1 - 0.49^5 = 97.18\%$\\

c) $1 - 0.51^5 = 96.55\%$\\

d) $0.49^5 + 0.51^5 = 6.28\%$\\

\vspace{20px}


\section{Satz von Bayes}
\textbf{V.)}\\
a)\\ 
1. 30\% bei 0.03\\
2. 70\% bei 0.05\\
3. Fehlsortierung: gesamt = $0.05 \cdot 0.7 + 0.03 \cdot 0.3 = 0.044$\\
4. Fehlsortierung von 2: $0.05 \cdot 0.7 = 0.035$

\[\frac{0.035}{0.044} \approx 0.7955 = 79.55\%\]

b)\\
$1-0.044 = 0.956 = 95.6\%$

% Matrix example
% \textbf{Korrektur:}\\
% $\mathbf{A} \odot \mathbf{B} =
% \begin{bmatrix}
%     1 & 1 & 1 \\
%     1 & 1 & 1 \\
%     1 & 0 & 1 \\
% \end{bmatrix}
% $

% tabular example 3 columns
% \renewcommand{\arraystretch}{1.5}
% \begin{center}
%     \begin{tabular}{ | m{12em} | m{12em} | m{12em} | }
%         \hline
%         1 & 2 & 3\\ 
%         \hline
%         1 & 2 & 3\\ 
%         \hline
%         1 & 2 & 3\\ 
%         \hline
%     \end{tabular}
% \end{center}


% tabular example 2 columns
% \renewcommand{\arraystretch}{1.5}
% \begin{center}
%     \begin{tabular}{ | m{17em} | m{17em} | }
%         \hline
%         1 & 2\\ 
%         \hline
%         1 & 2\\ 
%         \hline
%         1 & 2\\ 
%         \hline
%     \end{tabular}
% \end{center}

% \bibliography{}

\end{document}