\documentclass[12pt]{scrartcl}
\usepackage[ngerman]{babel}


\usepackage{amsmath, amssymb}

\usepackage{array}  % for the tables

\usepackage{nameref}  % for referencing with name

\usepackage{hyperref}  % for hyperlinks

\usepackage{mathrsfs}

\usepackage{graphicx}  % for the images

\usepackage{xcolor, colortbl}

\usepackage{gensymb} % for \degree

\usepackage{pgfplots}

\usepackage{tabto}

\newcolumntype{P}[1]{>{\centering\arraybackslash}p{#1}}

\usetikzlibrary{arrows}

% \usepgfplotslibrary{external}

% \tikzexternalize

\definecolor{Gray}{gray}{0.85}

\setlength{\parindent}{0cm}

% hyperlinks
\hypersetup{
    colorlinks,
    citecolor=black,
    filecolor=black,
    linkcolor=black,
    urlcolor=black
}

\bibliographystyle{IEEetran}




\author{David Jäggli}

\title{Diskrete Mathematik}



% ---------- Begin Main Document ----------- %



\begin{document}

\maketitle

\tableofcontents

\newpage
\section{Allg}

\subsection{Grundlagen der Logik und Beweise}

\begin{itemize}
    \item Die Regeln der Logik geben mathematischen Aussagen eine präzise Bedeutung.
    \item Konstruktion korrekter mathematischer Argumente
\end{itemize}

\subsection{Aussagen (Propositionen)}
\textbf{Propositionen:}
\begin{itemize}
    \item Bern ist die Bundesstadt
    \item 1 + 1 = 2
    \item Goldbachsche Vermutung: sie ist entweder wahr oder falsch, man weis es noch nicht
\end{itemize}

\textbf{Keine Propositionen:}
\begin{itemize}
    \item Wie spät ist es?
    \item x + 1 = 2
    \item Dieser Satz ist falsch.
\end{itemize}

Begründung: Es handelt sich hier nicht um Aussagen, die entweder wahr oder falsch sind.
Eine Aussage ist wahrheitsdefiniert. In einer Aussage darf nicht offen sein ob die Aussage wahr oder 
falsch sein kann. Sie darf sich auch nicht selbst widersprechen.

% ------------------------------------------------------
\section{Operatoren}

\begin{itemize}
    \item Negotiationsoperator: $\lnot$
    \item Konjunktion $\land$
    \item Disjunktion $\lor$
    \item Implikation $\rightarrow$
    \item Bikonditional $\leftrightarrow$
\end{itemize}



\subsection{Diskunktion}
$p \lor q$\\
Wenn p oder q wahr ist, ist die Aussage wahr (logic OR).


\renewcommand{\arraystretch}{1.5}
\begin{tabular}{ | m{3em} | m{3em} | m{3em} | }
    \hline
    p & q & $p \lor q$\\ 
    \hline
    w & w & w\\ 
    \hline
    w & f & w\\ 
    \hline
    f & w & w\\ 
    \hline
    f & f & f\\ 
    \hline
\end{tabular}


\subsection{Implikation}
$p \rightarrow q$\\
Wenn p dann q


\renewcommand{\arraystretch}{1.5}
\begin{tabular}{ | m{3em} | m{3em} | m{3em} | }
    \hline
    p & q & $p \rightarrow q$\\ 
    \hline
    w & w & w\\ 
    \hline
    w & f & f\\ 
    \hline
    f & w & w\\ 
    \hline
    f & f & w\\ 
    \hline
\end{tabular}


\subsection{Bikonditional}
$p \leftrightarrow q$\\
Wenn beide den gleichen Wahrheitswert haben ist die Aussage wahr.\\
\textbf{Wahrheitstabelle:}\\

\renewcommand{\arraystretch}{1.5}
\begin{tabular}{ | m{3em} | m{3em} | m{3em} | }
    \hline
    p & q & \(p \leftrightarrow q\)\\ 
    \hline
    w & w & w\\ 
    \hline
    w & f & f\\ 
    \hline
    f & w & f\\ 
    \hline
    f & f & w\\ 
    \hline
\end{tabular}

\subsection{Prioritäten}
\renewcommand{\arraystretch}{1.5}
\begin{tabular}{ | m{4em} | m{4em} | }
    \hline
    Operator & Priorität\\ 
    \hline
    $\lnot$ & 1\\ 
    \hline
    $\land$ & 2\\ 
    \hline
    $\lor$ & 2\\ 
    \hline
    $\rightarrow$ & 3\\ 
    \hline
    $\leftrightarrow$ & 3\\ 
    \hline
\end{tabular}


\section{Aussagen}

\subsection{Tautologie und Wiederspruch}
Tautologie ist eine Aussage, welche immer wahr ist.\\
Ein Wiederspruch ist eine Aussage, welche immer falsch ist.

\subsection{Logische Äquivalenzen}
Die Aussage p und q heissen logisch äquivalent, falls \(p \leftrightarrow q\) eine Tautologie ist. 
Man schreibt dann \(p \Leftrightarrow q\) oder
\(p \equiv q\) bzw. $p \sim q$

\subsection{Logische Äquivalenzregeln}
\includegraphics[width=14cm]{img/logic_equivalence_rules.png}

Weiterführend:\\
$p \rightarrow q \equiv \lnot p \lor q$


\newpage
\textbf{Beispiel angewandte logische Äquivalenzregeln}\\
Beispiel 1:\\
$\left(p \lor \lnot(q \land p)\right) \land \left(r \lor (s \lor r)\right)$\\
$\equiv (p \lor \lnot q \lor \lnot p) \land (r \lor r \lor s)$\\
$\equiv (T \lor \lnot q) \land (r \lor s)$\\
$\equiv T \land (r \lor s)$\\
$\equiv r \lor s$
 \\
 \\
 \\

Beispiel 2:\\
$(a \rightarrow (b \rightarrow c)) \rightarrow ((a \rightarrow b)\rightarrow (a \rightarrow c))$\\
$\equiv (a \rightarrow (\lnot b \lor c)) \rightarrow ((\lnot a \lor b) \rightarrow (\lnot a \lor c))$\\
$\equiv (\lnot a \lor (\lnot b \lor c)) \rightarrow (\lnot(\lnot a \lor b) \lor (\lnot a \lor c))$\\
$\equiv (\lnot a \lor \lnot b \lor c) \rightarrow ((a \land \lnot b) \lor \lnot a \lor c)$\\
$\equiv (\lnot a \lor \lnot b \lor c) \rightarrow ((a \lor \lnot a) \land (\lnot b \lor \lnot a) \lor c)$\\
$\equiv (\lnot a \lor \lnot b \lor c) \rightarrow (\lnot b \lor \lnot a \lor c)$\\
$\equiv$ X $\rightarrow$ X\\
$\equiv \lnot$X $\lor$ X\\
$\equiv$ T\\

\section{Quantoren}
Wird ein Quantor auf die Variable x angewandt, dann nennt man diese Variable \textit{gebunden}, ansonsten \textit{frei}.
\subsection{Prädikate}
Ein Prädikat ist ein Wortkonstrukt, welches mindestens eine Variable enthält.\\
$P(x) =$ ''$x > 3$''\\
Die Aussage $P(4) = 4 > 3$ ist wahr, während $P(2) = 2 > 3$ falsch ist.

\subsection{Allquantor}
Ist $P(x)$ wahr für alle x aus einer bestimmten Universalmenge, dann schreibt man $\forall x P(x)$.
Gelesen wird dies, "für alle $x$ gilt $P(x)$".\\
Falls es nur auf eine Bestimmte Zahlenmenge zutrifft (z.B. $\mathbb{Z}$) dann schreibt man:\\
$\forall x \in \mathbb{Z}$ ist wahr.

\subsection{Existenzquantor}
Ist $P(x)$ wahr für mindestens ein $x$ aus einer bestimmten Universalmenge, dann schreibt man $\exists x P(x)$
und liest: "es existiert ein $x$ für welches $P(x)$ wahr ist".


\subsection{Verschachtelte Quantoren}
Die Reihenfolge der Quantoren ist wesentlich; ausser alle Quantoren sind vom gleichen Typ (also Allquantoren oder Existenzquantoren)!


\newpage
\section{Beweise}
\begin{itemize}
    \item Ein Satz (Theorem) ist eine Aussage, von der man zeigen kann, dass sie wahr ist.
    \item Um zu zeigen, dass ein Satz wahr ist, verwendet man eine Abfolge (Sequenz) von Aussagen, die zusammen ein Argument, genannt Beweis ergeben.
    \item Aussagen können Axiome oder Postulate enthalten (grundlegende Annahmen der mathematischen Strukturen).
    \item Durch logisches (also gewissen Regeln gehorchendes) schliessen werden Folgerungen gemacht, die zusammen den Beweis ergeben.
    \item Ein Lemma ist ein einfacher Satz, der in Beweisen von komplizierteren Sätzen verwendet wird.
    \item Ein Korollar ist eine einfache Folgerung eines Satzes.
\end{itemize}


% ----------- Mengen ----------- %

\newpage
\section{Mengen}

Eine Menge ist eine ungeordnete Zusammenfassung wohldefinierter, unterscheidbarer
Objekte, genannt \textit{Elemente}, zu einem Ganzen. Für irgendein Objekt $x$ 
gilt dann bezüglich der Menge $A$ entweder $x \in A$ oder dann $x \notin A$.

\textbf{Beispiel:}\\
Endliche Mengen lassen sich durch Aufschreiben der in ihnen enthaltenen Elemente beschreiben.
z.B. die Menge aller natürlichen Zahlen kleiner als 101:\\
$A = {0, 1, 2, ..., 99, 100}$ (aufzählend notiert)\\
$99 \in A$ aber $101 \notin A$ (beschreibend notiert)\\

andere Schreibweisen sind: \\

$A = {n \in \mathbb{N} | n < 101} = {n \in \mathbb{N} : n <= 100} = {n|n \in \mathbb{N} \land n <= 100}$\\

\subsection{Gleichheit, elementare Mengen}
Zwei Mengen $A$ und $B$ sind \textbf{gleich} ($A = B$), falls sie dieselben Elemente enthalten.
$(A \subset B) \land (B \subset A)$\\

\textbf{Einige bekannte Mengen:}\\
$\mathbb{N}$ - \quad Menge der natürlichen Zahlen $(\mathbb{N}^* = \mathbb{N} \setminus \{0\})$\\
$\mathbb{Z}$ - \quad Menge der ganzen Zahlen\\
$\mathbb{Z}^+$ - \quad Menge der positiven ganzen Zahlen\\
$\mathbb{Q}$ - \quad Menge der Brüche\\
$\mathbb{R}$ - \quad Menge der reellen Zahlen\\
$\mathbb{C}$ - \quad Menge der komplexen Zahlen\\

\subsection{Spezielle Mengen}
\textbf{Teilmenge:} $A$ ist Teilmenge von $B$, geschrieben $A \subset B$, genau dann, wenn
$\forall x (x \in A \rightarrow x \in B)$: es gilt $A \subset A$!\\

\textbf{Leere Menge:} Für jede Menge $A$ gilt: $\emptyset \subset A$.\\

\textbf{Kardinalität:} Ist $S$ eine endliche Menge, dann bezeichnet $|S|$ die Kardinalität. Die
Kardinalität ist die Anzahl Elemente von $S$.\\

\textbf{Potenzmenge}: Die Potenzmenge $P(S)$ oder $2^S$ der Menge $S$ besteht aus der Menge aller
Teilmengen $A \subset S$.\\

\textbf{Beispiel:}\\
Bestimmen Sie die Potenzmenge von $S= \{1,2\}$\\
$S = \{1,2\}$\\
\quad $P(S) = 2^S = \{\emptyset, \{1\}, \{2\}, \{1, 2\}\}$\\
Es gilt allgemein $|2^S| = 2^{|S|}$


\subsection{Das Kreuzprodukt zweier Mengen / kartesisches Produkt}
$ A \times B = \{(a, b)|a \in A \land b \in B\}$\\
Reihenfolge ist entscheidend, $A \times B \neq B \times A$\\
$|A \times B| = |A| \cdot |B|$\\

\textbf{Beispiel:}
$A \times B = \{(1,a), (2,a), (3,a), (1,b), (2,b), (3, b)\}$


\subsection{Mengenoperationen}
\subsubsection{Komplement}
Ist $A$ eine Teilmenge der Menge $M$, so bezeichnet\\

$A^c = \overline{A} =  \{m \in M | m \notin A\}$\\

das Komplement von $A$ bezüglich $M$.

\subsubsection{Durchschnitt}
Sind $A$ und $B$ Teilmengen einer Menge $M$, so bezeichnet\\

$A \cap B = \{m \in M | m \in A \land m \in B\}$\\

den  Durchschnitt von $A$ und $B$.


\subsubsection{Vereinigung}
Sind $A$ und $B$ Teilmengen einer Menge $M$, so bezeichnet\\

$A \cup B = \{m \in M | m \in A \lor m \in B\}$\\

die Vereinigung von $A$ und $B$.


\newpage
\subsubsection{Differenz}
Sind $A$ und $B$ Teilmengen einer Menge $M$, so bezeichnet \\

$B \setminus A = \{m \in M | m \in B \land m \notin A\}$\\

die Differenz


\subsection{Set Operatoren}
\renewcommand{\arraystretch}{1.5}
\begin{tabular}{ | m{7em} | m{7em} | }
    \hline
    Allg. Operator & Set Operator \\ 
    \hline
    $p \lor q$ & $A \cup B$ \\ 
    \hline
    $p \land q$ & $A \cap B$ \\ 
    \hline
    $\lnot p$ & $\overline{A} $ \\ 
    \hline
\end{tabular}

\subsubsection{Rechenregeln}
\includegraphics[width=15cm]{img/Rechenregeln_mit_mengen.png}

\subsubsection{Mengen Identitäten}
\includegraphics[width=15cm]{img/set_identities.png }




\section{Funktionen}
Wird jedem Element $x$ einer Menge $X$ genau ein Element $y$ einer Menge
$Y$ zugeordnet, so heisst die Zuordnung \textbf{Funktion}.


\subsection{Die ceiling- und floorfunction}
$\lceil \cdot \rceil$ :  $\mathbb{R} \rightarrow \mathbb{Z}$, $x \mapsto \lceil x \rceil =$ min$\{n \in \mathbb{Z} | x \leq n\}$\\
$\lfloor \cdot \rfloor$ :  $\mathbb{R} \rightarrow \mathbb{Z}$, $x \mapsto \lfloor x \rfloor =$ max$\{n \in \mathbb{Z} | n \leq x\}$\\


\subsection{Injektive Funktionen}
Eine Funktion heisst injektiv, wenn jedes $x$ auf eine eigenes $y$ zeigt.

\subsection{Surjektive Funktionen}
Eine Funktion heisst surjektiv, falls für jedes Element $y$ ein Element $x$ existiert, so dass $f(x) = y$ gilt.

\subsection{Bijektive Funktionen}
Eine Funktion heisst bijektiv, falls sie injektiv und surjektiv ist. Das bedeutet, dass jedes Element $y$ genau
ein zugehöriges Element $x$ hat.\\

Bijektive Funktionen sind umkehrbar. Man muss einfach die Pfeile umkehren und damit entsteht
aus $f$ die Umkehrfunktion $f^{-1}$.


\subsection{Zusammengesetzte Funktionen}
Gegeben seien zwei Funktionen, so dass der Wertebereich von $g$ im Definitionsbereich von $f$ enthalten ist.
Dann kann man die so genannte \textbf{zusammengesetzte Funktion} oder \textbf{Komposition} von $f$
und $g$ bilden:\\
$F = f \circ g$ : $X \longmapsto Y$, $x \longmapsto f(g(x))$


\subsection{Die Caesar-Chiffre}
\begin{enumerate}
    \item \textbf{Kodierung:} Buchstaben auf Zahlen abbilden\\
    K:\{$a$,$b$,$c$,...,$z$\} $\mapsto$ \{0,1,2,...,25\}, wobei $a \mapsto 0$,$b \mapsto 1$,$c \mapsto 2$,$z \mapsto 25$
    \item \textbf{Verschlüsseln:} die eigentliche Caesar-Verschlüsselung\\
    V:\{0,1,2,...,25\} $\mapsto$ \{0,1,2,...,25\}, $m \mapsto c := (m + 3)$ mod 26.
    \item \textbf{Dekodierung:} Zahlen auf Buchstaben abbilden\\
    D:\{0,1,2,...,25\} $\mapsto$ \{0,1,2,...,25\}, wobei $0 \mapsto a$,$1 \mapsto b$,$2 \mapsto c$,$25 \mapsto z$
\end{enumerate}


\subsection{Umkehrfunktionen}
Wenn man die Umkehrfunktion auf das Ergebnis der Ursprungsfunktion mit einem $x$-Wert anwendet
erhält man wieder $x$. Heisst:\\
$f^{-1}(f(x)) = x$


\newpage
\section{Folgen}
\subsection{Definition}
Eine \textbf{Folge} ist eine Abbildung von $\mathbb{N}$ (oder auch $\mathbb{N}^* = \mathbb{N} \setminus \{0\}$) in eine Menga $A$:\\
$\{\cdot\}$:$\mathbb{N} \mapsto A$, $n \mapsto a_n$\\
Man nennt $a_n$ das Glied der Folge mit der Nummer $n$. Die Folge wird auch mit \{$a_n$\} oder ($a_n$) bezeichnet.\\

\textbf{Example:}\\
Man schreibe die ersten sechs Glieder der Folge auf, deren k. Glied gegeben ist durch $a_k = \frac{1}{k}$. \\

$a_k = \left(1,  \dfrac{1}{2}, \dfrac{1}{3}, \dfrac{1}{4}, \dfrac{1}{5}\dots\right)$


\subsection{Die geometrische Folge}
Bei einer geometrischen Folge ist der Quotient zweier aufeinander folgender Glieder immer
gleich, nämlich q. Das bedeutet, dass $\frac{a_{k+1}}{a_k}$ immer gleich ist.


\subsection{Summen}
Dank Summenzeichen lassen sich Summen einfacher schreiben:
\[\sum_{j=m}^{n} a_j = a_m + a_{m+1} + a_{m+2} + \dots + a_n\]

\[\sum_{j=m}^{n} a_j = \sum_{i=0}^{n-m} a_{m+i} = \sum_{k=1}^{n-m+1} a_{m+k-1}\]


Addiert man die Glieder einer arithmetischen Folge ($a_k$), entsteht die \textbf{arithmetische Reihe:}

\[\sum_{k=0}^{n-1} a_k = n \frac{a_0 + a_{n-1}}{2}\]


\newpage
\textbf{Nützliche Summenformeln:}

\renewcommand{\arraystretch}{2.5}
\begin{center}
    \begin{tabular}{  m{10em} | P{10em} }
        \hline
        Summe & geschlossene Form \\
        \hline
        $\sum_{k=0}^{n} x^k$                        & $\frac{x^{n + 1} - 1}{x - 1}$               \\ 
        $\sum_{k=0}^{n} 2^k$                        & $2^{k + 1} - 1$               \\ 
        $\sum_{k=1}^{n} k$                          & $\dfrac{n(n + 1)}{2}$         \\ 
        $\sum_{k=1}^{n} k^2$                        & $\dfrac{n(n + 1)(2n + 1)}{6}$ \\ 
        $\sum_{k=1}^{n} k^3$                        & $\dfrac{n^2(n + 1)^2}{4}$     \\ 
        $\sum_{k=0}^{\infty} x^k$, $|x| < 1$        & $\dfrac{1}{1 - x}$            \\ 
        $\sum_{k=1}^{\infty} kx^{k-1}$, $|x| < 1$   & $\dfrac{1}{(1-x)^2}$          \\ 
        \hline
    \end{tabular}
\end{center}


\subsection{Produkte}
Dank dem Produktzeichen lassen sich Produkte einfacher schreiben:\\


\[a_m \cdot a_{m+1} \cdot a_{m+2} \dots a_n = \prod_{j=m}^{n} a_j \quad\quad n \geqslant m\]

Die Fakultät lässt sich mithilfe des Produktzeichens wie folgt schreiben:\\
\[n! = 
\begin{cases}
    1 & n=0 \\
    n(n - 1)(n - 2) \dots 2 \cdot 1 = \prod_{k=1}^{n}k & n > 0
\end{cases}\]


Nützliche Abkürzung:
\[\prod_{i=1}^{n} i = \frac{n \cdot (n + 1)}{2}\]


\newpage
\section{Algorithmen}

Ein Algorithmus ist eine endliche Menge von präzisen Instruktionen mit deren Hilfe
eine Berechnung ausgeführt oder ein Problem gelöst wird.

\textbf{Algorithmen haben folgende Eigenschaften:}
\begin{enumerate}
    \item einen genau spezifizierten Input und daraus berechneten Output
    \item die Instruktionen sind präzise, korrekt für jeden möglichen Input und in endlicher Zeit durchführbar
\end{enumerate}


Greedy Algorithmen wählen in jedem Schritt, die zu diesem Zeitpunkt die effizienteste ist.


\newpage
\section{Wachstum von Funktionen}

\subsection{Definition}
Seien $f$ und $g$ Funktion von $\mathbb{Z}$ oder ($\mathbb{R}$). Dann sagt man "$f(x)$ ist $\mathcal{O}(g(x))$", falls es
Konstanten $C$ und $k$ gibt, so dass gilt:

$|f(x)| \leq C|g(x)|$, $\forall x > k$ Lies: "f(x) ist gross-O von g(x), man schreibt: $f(x) \in \mathcal{O}(g(x))$.

\begin{itemize}
    \item Meist ist f eine komplizierte Funktion, wie z.B. $f(x) = (x^2 + 1) ln x + (2^x + x^4)$
    \item Man möchte für g eine möglichst einfache, nicht zu schnell wachsende Funktion, wie z.B. $x$, $x^2$ \dots
    \item Ziel ist es herauszufinden, wie sich f(x) für sehr, sehr grosse $x$ verhält, und zwar verglichen mit der einfacheren Funktion $g$.
    \item k ist der kleinste Wert von $x$, für den die obige Ungleichung noch gilt!
\end{itemize}

Also wir wollen für sehr grosse $x$, eine einfachere Funktion zu finden.


\subsection{Example}
Für $f(x) = x^2 + 2x + 1$ ist $\mathcal{O}(x^2)$.\\
Das heisst bei sehr grossen $x$ entspricht die Funktion $f(x) =  x^2$

\includegraphics[width=15cm]{img/wachstum_example_1.png}
\newpage
\includegraphics[width=15cm]{img/wachstum_example_2.png}


\subsection{Polynome}
Für das Polynom $\sum_{k=0}^n a_k x^k$ gilt $f(x)$ ist $\mathcal{O}(x^n)$. Das heisst
die höchste Potenz von x gibt den Ton an.\\

\textbf{Beispiel:}\\
Es gilt immer: $|a + b| \leq |a| + |b|$\\

$f(x) = 5x^6 - 3x^2 + x - 10$\\
$|f(x)| \leqslant 5x^6 + 3x^2 + x + 10$\\
$|f(x)| \leqslant 5x^6 + 3x^6 + x^6 + 10x^6$\\
$|f(x)| \leqslant 5x^6 + 3x^6 + x^6 + 10x^6$ für $x \geqslant 1$\\
$|f(x)| = 19x^6$\\

also $f$ ist $\mathcal{O}(x^6)$ mit Zeugen $k=1$ und $C = 19$


\newpage
\section{Zahlen und Division}
\subsection{Definition}
Falls $a$, $b \in \mathbb{Z}$ mit $a \neq 0$ dann sagt man: $a$ \textit{teilt} $b$, falls $\exists c (b = ac)$
in der Universalmenge $\mathbb{Z}$. Dann ist $a$ ein \textit{Faktor} von $b$ und $b$ ein \textit{Vielfaches}
von $a$. Man schreibt dann $a \mid b$ und anderenfalls $a \nmid b$\\

\textbf{Theorem:}\\
Falls a, b, c $\in \mathbb{Z}$\\
(a) $a \mid  b  \land a \mid c \rightarrow a \mid (b + c)$, $\rightarrow 6 \mid 12 \land 6 \mid 24 \rightarrow 6 \mid (12+24)$ \\  
(b) $a \mid b  \rightarrow \forall c(a \mid bc) $,\\  
(c) $a \mid b  \land b \mid c \rightarrow a \mid c$,\\  


\subsection{ggt kgV}
Der ggT von a und b beschreibt das grösste $d$ für welches gilt $d \mid a$ und $d \mid b$.\\
Zwei zahlen sind teilerfremd (relaitv prim) falls ggT(a,b) = 1, dann schreibt man
$a \perp b$. \\

Das kgV zweier Zahlen $a$ und $b$ ist die kleinste positive Zahl, welche durch
$a$ und $b$ teilbar ist. Es gilt:\\

ab = ggT(a,b) $\cdot$ kgV(a,b)\\

\textbf{Für ggT finden:}
\begin{enumerate}
    \item a und b jeweils in Primfaktoren zerlegen
    \item alle gemeinsamen Primfaktoren multiplizieren
\end{enumerate}


\subsection{Modulare Arithmetik}
Sei $m \in \mathbb{N} \backslash \{0\}$, dann nennt man zwei ganze Zahlen a und b kongruent modulo m, falls $m\mid (a - b)$
Das heisst a und b liegen ein Vielfaches von m auseinander. Man schreibt dann $a \equiv b$ mod $m$ und sagt:
"a ist kongruent zu b modulo m".\\

$13 \equiv 1$ mod 4 denn $4 \mid (13 - 1)$\\
$13 \equiv 1$ mod 3 denn $3 \mid (13 - 1)$\\
$13 \not\equiv  1$ mod 5 denn $5 \nmid (13 - 1)$\\


\newpage
\subsection{Der Euklidische Algorithmus}
Effiziente Methode um ggT zu finden.

\includegraphics[width=15cm]{img/euqlidic_algorithm.png}

ggT ist jeweils 1 und 3.

\section{Matrizen}
\subsection{Definition}
Eine m $\times$ n-Matrix ist eine rechteckige Anordnung von Zahlen in m Zeilen und n
Spalten.\\

\renewcommand{\arraystretch}{1}
\textbf{A} = 
$
\begin{bmatrix}
    a_{1,1} & a_{1,2} & \cdots & a_{1,n}\\
    a_{2,1} & a_{2,2} & \cdots & a_{2,n}\\
    \vdots & \vdots & \ddots & \vdots\\
    a_{m,1} & a_{m,2} & \cdots & a_{m,n}
\end{bmatrix}$\\


Kurzschreibform: \textbf{A} = [$a_{i,j}$]


\subsection{Addition von Matrizen}

Addition von Matrizen erfolgt jeweils durch die Addition der einzelnen Positionen


\subsection{Multiplikation mit einer Zahl}
Einfach jede Zahl multiplizieren.


\subsection{Matrixmultiplikation}
\textbf{C = AB}, wobei die Anzahl Spalten in \textbf{A} gleich der Anzahl Reihen in \textbf{B}
sein muss

\includegraphics[width=15cm]{img/matrizenmultiplikation.png}

\subsection{Transporierte Matrix}
Eine transponierte Matrix ist eine, bei der die Spalten und Reihen vertauscht wurden.\\

\includegraphics[width=15cm]{img/transponierte_matrix.png}

\subsection{Matrizen Eigenschaften}
Keywords: symmetrisch, antisymmetrisch, Einheitsmatrix, k-te Potenz

\includegraphics[width=16cm]{img/matrizen_eigenschaften.png}

TODO: (SW03) Inverse Matrix und Matrizen Eigenschaften allgemein \& Rechenregeln mit Matrizen.

\subsection{Null-Eins Matrizen}
Auch boolesche Matrizen genannt.\\

Boolesches Matrizen Produkt wird folgendermassen geschrieben: $\mathbf{A} \odot  \mathbf{B}$.\\

\includegraphics[width=15cm]{img/boolesches_produkt.png}

\newpage
Eine quadratische Matrix kann auch eine Potenz haben:\\
$\mathbf{A}^{[r]} = \mathbf{A} \odot \mathbf{A} \dots \odot \mathbf{A}$:\\

\renewcommand{\arraystretch}{1}
$\mathbf{A} = 
\begin{bmatrix}
    0 & 0 & 1\\
    1 & 0 & 0\\
    1 & 1 & 0
\end{bmatrix}\quad\quad \mathbf{A}^2 = \mathbf{A} \odot \mathbf{A} = 
\begin{bmatrix}
    0 & 0 & 1\\
    1 & 0 & 0\\
    1 & 1 & 0
\end{bmatrix} 
\odot
\begin{bmatrix}
    0 & 0 & 1\\
    1 & 0 & 0\\
    1 & 1 & 0
\end{bmatrix}$

\newpage
\section{Mathematisches Begründen}
Bekannte Beweismethoden:
\begin{itemize}
    \item \textbf{Direkter Beweis:} Man zeigt, dass $p \rightarrow q$ wahr ist.
    \item \textbf{Beweis durch Kontraposition:} Man verwendet, dass $p \rightarrow q$ äquivalent ist zur Kontraposition
            $\lnot q \rightarrow \lnot q$.
    \item \textbf{Beweis durch Widerspruch:} Wir möchten zeigen, dass $p$ wahr ist indem\dots
\end{itemize}


\subsection{Mathematische Induktion}

\begin{enumerate}
    \item \textbf{Induktionsverankerung:} Für die kleinste Zahl zeigen, dass die Formel wahr ist (1 bei $n \in \mathbb{N}$).
    \item \textbf{Induktionsschritt:} Es wird gezeigt, dass die Implikation $P(k) \rightarrow P(k+1)$ wahr ist $\forall k \geq 1$.
\end{enumerate}

Beispiel: Dominosteine $\rightarrow$ falls der erste fällt, muss der 2. auch fallen. Falls der 2.
fällt muss der 3. auch fallen.\\

\textbf{Hinweis:} Immer zuerst überlegen was am Schluss herauskommen sollte, falls der Beweis mit Induktion
bewiesen werden kann, dann fällt auch das Beweisen leichter.


\subsection{Rekursiv definierte Funktionen}
Wenn eine Funktion mit Definitionsbereich $D(f) = \mathbb{N}$ für die f(0) definiert ist und
bei welcher f(k) durch f(k-1), f(k-2) \dots f(1), f(0) berechnet wird.
\textbf{Beispiel:} Fibonacci Folge.\\

Diese kann man auch mit Induktion beweisen.


\subsection{Beispiel Türme von Hanoi}
Vermutung: $f(n) = 2^n - 1$\\
Das heisst $f(n+1) = 2^{n+1} - 1$\\


$f(1) = 2 - 1 = 1$: stimmt\\
Es braucht $2^n$ Züge um einen Turm zu bewegen.\\
Dann brauch es +1 um die unterste Scheibe (n+1 Scheibe) zu verschieben.\\
Und schlussendlich noch einmal $2^n + 1$\\
Das ergibt: $2* (2^n-1) + 1 = 2^{n+1} - 2 + 1 = 2^{n+1} - 1$ \\
\underline{Vermutung stimmt.}

\section{Grundlagen des Zählens}
\subsection{Zusammenfassung}
\includegraphics[width=15cm]{img/wahrscheinlichkeiten.png}


\subsection{Schubfachprinzip}
Es gibt wenigstens ein Fach in das mehr als 2 Objekte reingehen.\\

\textbf{Beispiel:} In jeder Menge von 5 Zahlen gibt es 2, welche bei einer
Division durch 4 den gleichen Rest geben.\\
Bei einer Divison durch 4 gibt es Reste von 0, 1, 2 oder 3. Man hat 5 Zahlen, heisst
2 Zahlen müssen sich denselben Rest teilen.\\


\subsection{Permutationen}
Eine Permutation von $n$ verschiedenen Elementen ist eine geordnete Anordnung dieser $n$ Elemente.\\
Das heisst die Anordnung (3,1,2) der Menge S={1,2,3} ist eine Permutation von $S$.\\
3-Permutationen ((1,2,3), (1,3,2), (2,1,3), (2,3,1), (3,1,2), (3,2,1)): 3!\\
2-Permutationen ((1,2), (1,3), (2,1), (2,3), (3,1), (3,2)): $2 \cdot 3$\\

Allgemeine Formel für Anzahl r-Permutationen einer Menge von n Elementen:\\

\[P(n, r) = \frac{n!}{(n-r)!}\text{, } \quad 0 \leq r \leq n \in \mathbb{N}\]

n = die Anzahl Elemente\\
r = die Anzahl Elemente im Tuple\\

Die Reihenfolge \textbf{spielt} eine Rolle.


\newpage
\subsection{Permutation nicht unterscheidbarer Objekte}
Die Anzahl verschiedener Permutationen von n Objekten, von denen $n_1$ Objekte der Art 1, $n_2$ Objekte der Art 2, 
\dots, $n_k$ Objekte der Art k sind, ist gegeben durch:

\[\frac{n!}{n_1!n_2! \dots n_k!} \text{, wobei } n = \sum_{i=1}^{k} n_i\]

\textbf{Beispiel:}\\
wie viele Wörter kann man aus den Zeichen von SUCCESS machen?\\
$n = SUCCESS = 7$\\
$n_1 = S = 3$\\
$n_2 = U = 1$\\
$n_3 = C = 1$\\
$n_4 = E = 2$\\

Ergibt: $\displaystyle{\frac{7!}{3!2!1!1!} = \frac{7!}{3 \cdot 2 \cdot 2} = 420}$

\subsection{Kombinationen}

Für $S = \{1, 2, 3, 4\}$ ist \{1, 3, 4\} eine 3-Kombination von S. Beachte, dass \{3, 1, 4\} die selbe
3-Kombination von S ist.\\

Die Reihenfolge spielt \textbf{keine} Rolle.\\

Die Anzahl von $r$-Kombinationen einer Menge von $n \geq 0$ Elementen isst gegeben durch:

\[C(n, r) = \frac{n!}{r!(n-r)!} = \binom{n}{r} = C(n, n-r)\]

n = die Anzahl Elemente\\
r = die Anzahl Elemente im Set\\




\subsection{Kombinationen mit Wiederholungen}
\textbf{Beispiel:} Wie viele verschiedene Früchteschalen kann man mit Äpfeln, Orangen und Birnen
machen, wenn immer 4 Früchte verwendet werden?\\
AAAA, AAAO, AAAB, AAOO, AAOB \dots\\

\[C(n+r-1, r) = \binom{n+r-1}{r}\]


\section{Diskrete Wahrscheinlichkeitsrechnung}
Wahrscheinlichkeiten werden oft mit $p(A)$ angegeben. Wobei $p$ die Wahrscheinlichkeit allgemein
beschreibt und $A$ der Output. Heisst $p(A)$ ist die Wahrscheinlichkeit mit der das Ereignis $A$ 
eintrifft.\\

\subsection{Bedingte Wahrscheinlichkeit}
\textbf{Definition:} Die Wahrscheinlichkeit, dass ein Ereignis A eintritt, wenn ein Ereignis B eingetreten ist, ist gegeben durch\\
$p(A|B) = \frac{p(A \cap B)}{p(B)}$ (siehe Beispiel mit Münze in SW06). 



\subsection{Verteilungsfunktionen}
\includegraphics[width=15cm]{img/zusammenfassung _verteilungen.png}
\subsubsection{Bernoulli-Verteilung}
Zufallsexperiment mit nur 2 möglichen Ergebnissen, wobei die Bernoulli-Verteilung der 
Wahrscheinlichkeitsverteilung entspricht. Wobei wahr p entspricht und falsch 1-p.
Die Wahrscheinlichkeiten müssen dabei voneinander unabhänig sein.\\

\textbf{Beispiel:}
Bitstring mit 3 Bits bei der n-Einsen vorkommen (und wenn man eine 1 Würfelt ist es ein
Erfolg resp. $p$ oder ein nicht-Erfolg resp. $1-p$ oder $q$).\\

P(k=0) = P(\{000\}) = $1p^0 \cdot (1-p)^3$\\
P(k=1) = P(\{001\},\{010\},\{010\}) = $3p^1 \cdot (1-p)^2$\\
P(k=2) = P(\{011\},\{110\},\{101\}) = $3p^2 \cdot (1-p)^1$\\
P(k=3) = P(\{111\}) = $1p^3 \cdot (1-p)^0$\\

\textbf{Definition durch Binomialverteilung:}\\
$B(k|n,p) = B_{n,p}(k) = C(n,k)p^k(1-p)^{n-k} = \binom{n}{k}p^k(1-p)^{n-k}$\\
B: Bernoulli\\
k: Anzahl Erfolge\\
n: Anzahl Versuche\\
p: Wahrscheinlichkeit für Erfolg\\

Mit dieser Formel kann man die Wahrscheinlichkeit ausrechnen für k-Erfolge.\\


\textbf{Beispiel für eine Ungleichverteilung:}\\
Eine 0 wird zu 90\% und eine 1 zu 10\% gewürfelt. Wie wahrscheinlich ist es 8 Nullen bei
10 Würfen zu erzielen.\\

$B(8|10,0.9) = \binom{10}{8} \cdot 0.9^8 \cdot 0.1^2 = 19.37\%$

\subsubsection{Hypergeometrische Verteilung}
Binomialverteilung liegt vor, wenn bei einer Stichprobe die Objekte wieder zurückgelegt werden.
Werden die Objekte nicht wieder zurückgelegt handelt es sich um die Hypergeometrische Verteilung.\\

\textbf{Beispiel:}\\
Wenn man insgesamt N Objekte hat und M davon sind defekt. Wie gross ist die Wahrscheinlichkeit,
dass von n gezogenen Objekten k Objekte defekt sind?\\

$\displaystyle{p(k) = \frac{\binom{M}{k} \binom{N-M}{n-k}}{\binom{N}{n}}}$\\
\vspace{1cm}


\textbf{Anschauliches Beispiel:}\\
\includegraphics[width=15cm]{img/hypergeometrische Verteilung.png}

Grün: Anzahl der Objekte\\
Rot: Anzahl der Objekte die defekt sind\\
Schwarz: Anzahl der Objekte die nicht defekt sind\\


\subsubsection{Poisson-Verteilung}
Wenn bei einer Binomialverteilung sehr viele Versuche durchgeführt werden und die Wahrscheinlichkeit 
sehr klein ist, ist die Annäherung durch die Poisson-Verteilung viel einfacher.\\
$p \rightarrow 0$ und $n \rightarrow \infty$ dann $\mu = np$:\\

$\displaystyle{p(k) = \frac{\mu^k e^{-\mu}}{k!}}$\\


\subsection{Zufallsvariablen}
Wahrscheinlichkeitsverteilung einer Zufallsvariable $X$ ist die Verteilungsfunktion $F_X(x)$,


\section{Fortgeschrittene Zählmethoden}
\subsection{Rekursionsbeziehungen}
Rekursionsbeziehungen liegen immer vor, wenn ein Wert $a_n$ von einem anderen Wert $a_{n-1}$ abhängt.\\
Beispiel für Rekursionsbeziehungen:\\
Wie viele Bitstrings der Länge $n$ gibt es, die keine zwei aufeinanderfolgende
Nullen enthalten?\\

$a_n = a_{n-1} + a_{n-2}$\\

Ein Bitstring endet mit einer 1 oder 0. Im ersten Fall kann irgendein Wert davor stehen, davon
gibt es $a_{n-1}$ Möglichkeiten. Im zweiten Fall muss ein 1 vor der 0 stehen, davon gibt es $a_{n-2}$ Möglichkeiten.\\


\textbf{Beispiel Computersystem Passwort:}\\
Ein Passwort ist korrekt, wenn es eine gerade Anzahl von Nullen hat.\\

$a_{n-1}$ Möglichkeiten bei einer Zahl, welche mit 1-9 endet.\\
Wenn eine Null steht, dann noch $10^{n-1} - a_{n-1}$ Möglichkeiten.\\

$a_n = 9 \cdot a_{n-1} + 10^{n-1} - a_{n-1}$\\
$a_n = 8 \cdot a_{n-1} + 10^{n-1}$\\


\subsection{Lösen von Rekursionsbeziehungen}
\textbf{Eigenschaften von Rekursionsbeziehungen:}\\
$F(a_n, a_{n-1} \dots a_2, a_1) = r(n)$\\

Sie ist homogen, falls $r(n) = 0$\\

Ordnungen von Rekursionsbeziehungen:\\
Sie ist k-ter Ordnung, falls F höchstens von Gliedern ab $a_{n-k}$ abhängt.



\section{Graphentheorie}
\subsection{Gewichtete Graphen}
Ein gewichteter Graph G hat eine Gewichtsfunktion\\
$w: E \rightarrow (0,\infty), \{u,v\} \rightarrow w(u,v)$,\\
welche jedem Kantenpaar ein Gewicht zuordnet.\\



% Matrix example
% $
% \begin{bmatrix}
%     a_{1,1} & a_{1,2} & a_{1,n}\\
%     a_{2,1} & a_{2,2} & a_{2,n}\\
%     a_{m,1} & a_{m,2} & a_{m,n}
% \end{bmatrix}$\\

% tabular example 3 columns
% \renewcommand{\arraystretch}{1.5}
% \begin{center}
%     \begin{tabular}{ | m{12em} | m{12em} | m{12em} | }
%         \hline
%         1 & 2 & 3\\ 
%         \hline
%         1 & 2 & 3\\ 
%         \hline
%         1 & 2 & 3\\ 
%         \hline
%     \end{tabular}
% \end{center}


% tabular example 2 columns
% \renewcommand{\arraystretch}{1.5}
% \begin{center}
%     \begin{tabular}{ | m{17em} | m{17em} | }
%         \hline
%         1 & 2\\ 
%         \hline
%         1 & 2\\ 
%         \hline
%         1 & 2\\ 
%         \hline
%     \end{tabular}
% \end{center}


% \begin{tikzpicture}[line cap=round,line join=round,>=triangle 45,x=0.5cm,y=0.25cm]
%     \begin{axis}[
%     x=0.75cm,y=0.5cm, % size of the grid
%     axis lines=middle,
%     ymajorgrids=true,
%     xmajorgrids=true,
%     xmin=-10,
%     xmax=10,
%     ymin=-10,
%     ymax=10,
%     xtick={-11,-10,...,10},
%     ytick={-10,-9,...,9},]
%     \draw[line width=2pt,color=blue] (-10,-5) -- (-2,-1);
%     \begin{scriptsize}
%         \draw[color=blue] (-9.866,-4.728) node {$g$};
%         \draw[color=blue] (-1.906,7.172) node {$f$};
%         \draw[color=blue] (3.134,5.232) node {$h$};
%     \end{scriptsize}
% \end{axis}
% \end{tikzpicture}




% \bibliography{quantum_ready}

\end{document}