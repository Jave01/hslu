\documentclass[12pt]{scrartcl}
\usepackage[ngerman]{babel}


\usepackage{amsmath, amssymb}

\usepackage{array}  % for the tables

\usepackage{nameref}  % for referencing with name

\usepackage{hyperref}  % for hyperlinks

\usepackage{mathrsfs}

\usepackage{graphicx}  % for the images

\usepackage{xcolor, colortbl}

\usepackage{gensymb} % for \degree

\usepackage{pgfplots}

\usepackage{tabto}

\usepackage{ulem} % \uuline

\usetikzlibrary{arrows}

% \usepgfplotslibrary{external}

% \tikzexternalize

\definecolor{Gray}{gray}{0.85}

\setlength{\parindent}{0cm}

\newcommand{\RomanNumeralCaps}[1]
    {\MakeUppercase{\romannumeral #1}}

% hyperlinks
\hypersetup{
    colorlinks,
    citecolor=black,
    filecolor=black,
    linkcolor=black,
    urlcolor=black
}

\bibliographystyle{IEEetran}




\author{David Jäggli}

\title{Diskrete Mathematik - Übungen SW09}



% ---------- Begin Main Document ----------- %



\begin{document}

\maketitle

\tableofcontents

\newpage
\section{Einführung in die Zahlentheorie}
\textbf{I.)}\\
$n \geq 2$ und $n \in \mathbb{N}$\\

a) $n \mod n = 0$ dann muss $(n+1) \mod n = 1$\\
b) $n^2$ ist immer ein Vielfaches von $n$, heisst $n^2 \mod n = 0$\\
c) $(3n+6) \mod n = 3n \mod n + 6 \mod n = 6 \mod n$\\
d) $4n -1 \mod n = (4n \mod n) + (-1 \mod n) = -1 \mod n = n - 1 \mod n = n - 1$ \\
e) $(n^2 + n) \mod n = 0$\\
f) $(n^3 + 2n^2 + 4) \mod n = 4 \mod n$\\
g) $((2n + 2)(n + 1)) \mod n = (2n^2 + 4n + 2) \mod n = 2 \mod n$\\
h) $n! \mod n = n \cdot (n-1)! \mod n = 0$\\


\textbf{Korrektur:}\\
g)
$ 2 \mod n = 
    \begin{cases} 
    0               & \text{für } n = 2 \\
    2               & \text{für } n > 2\\
 \end{cases}
$


\vspace{1cm}
\textbf{II.)}\\
$ggT(587, 392) = $\\
\begin{align*}
    587 &= 1 \cdot 392 + 195 \\
    392 &= 2 \cdot 195 + 2 \\
    195 &= 97 \cdot 2 + 1 \\
    97  &= 97 \cdot 1 + 0  \\
\end{align*}

$ggT(587, 392) = 1$\\



\textbf{erweiterter euklidischer Algorithmus}

\begin{align*}
    1 &= 195 - 97 \cdot 2 \\
    &= 195 - 97 \cdot (392 - 2 \cdot 195) = 195 \cdot 195 -  97 \cdot 392\\
    &= 195 \cdot (587 - 392) - 97 \cdot 392\\
    &= 195 \cdot 587 - 195 \cdot 392 - 97 \cdot 392\\
    &= 195 \cdot 587 - 212 \cdot 392\\
\end{align*}\\

\newpage
Daraus folgt
\begin{align*}
    1 &\equiv 195 \cdot 587 - 292 \cdot 392 \text{ (mod } 587) \\
      &\equiv (587 - 292) \cdot 392 \text{ (mod } 587) \\
      &\equiv 295 \cdot 392 \text{ (mod } 587) \\
\end{align*}


\textbf{III.)}\\
Bestimme alle Lösungen von den Kongruenzen:
\begin{align*}
    x  &\equiv 1 \mod 2 \\
    x  &\equiv 2 \mod 3 \\
    x  &\equiv 3 \mod 5 \\
    x  &\equiv 4 \mod 11 \\
\end{align*}

Obvious:
\begin{align*}
    m = m_1 \cdot m_2 \cdot m_3 \cdot m_4 = 330 \\
    M_1 = \frac{m}{m_1} = 165 \\
    M_2 = \frac{m}{m_2} = 110 \\
    M_3 = \frac{m}{m_3} = 66 \\
    M_4 = \frac{m}{m_4} = 30 \\
\end{align*}

\newpage
Für alle $i(1 \leq i \leq 4)$ muss $M_i \cdot y_i$ berechnet werden.\\

$i_1$:\\
\begin{align*}
    165 &= 2 \cdot 82 + 1\\
    \\
    y_1 &= 1\\
\end{align*}

$i_2$:\\
\begin{align*}
    110 &= 3 \cdot 36 + 2\\
    \\
    y_2 &= 2\\
\end{align*}

$i_3$:\\
\begin{align*}
    66 &= 2 \cdot 82 + 1\\
    \\
    y_3 &= 1\\
\end{align*}

\newpage
$i_4$:\\
\begin{align*}
    30 &= 11 \cdot 2 + 8\\
    11 &= 1 \cdot 8 + 3\\
    8 &= 2 \cdot 3 + 2\\
    3 &= 1 \cdot 2 + 1\\
    \\
    1 &= 3 - 1 \cdot 2\\
      &= 3 - 1 \cdot (8 - 2 \cdot 3)\\
      &= 3 \cdot 3 - 8\\
      &= 3 \cdot (11 - 8) - 8\\
      &= 3 \cdot 11 - 4 (30 - 2 \cdot 11)\\
      &= 11 \cdot 11 - 4 \cdot 30\\
    \\
    y_4 &= -4 = 7\\
\end{align*}


Nach x auflösen:
\begin{align*}
    x &\equiv \sum_{i=1}^{4} r_i \cdot M_i \cdot y_1 \text{ (mod }m) \\
      &\equiv 1 \cdot 165 \cdot 1 + 2 \cdot 110 \cdot 2 + 3 \cdot 66 + 4 \cdot 30 \cdot 7 \text{ (mod }330) ) \\
      &\equiv 1643 \text{ (mod }330)\\
      &\equiv 323\\
\end{align*}


\textbf{IV.)}\\
$12! = 2 \cdot 3 \cdot 4 \cdot 5 \cdot 6 \cdot 7 \cdot 8 \cdot 9 \cdot 10 \cdot 11 \cdot 12$\\
$12! = 2 \cdot 3 \cdot 2^2 \cdot 5 \cdot 2 \cdot 3 \cdot 7 \cdot 2^3 \cdot 3^2 \cdot 2 \cdot 5 \cdot 11 \cdot 2^2 \cdot 3$\\
$12! = 2^{10} \cdot 3^5 \cdot 5^2 \cdot 7 \cdot 11$\\

$\phi(12!) = 2^9 (2 - 1) \cdot 3^4 (3-1)  \cdot 5 (5-1) \cdot (7-1) \cdot (11 - 1)$\\
$\phi(12!) = 2^9 \cdot 3^4 \cdot 2 \cdot 6 \cdot 4 \cdot 5 \cdot 10$\\
$\phi(12!) = 2^9 \cdot 3^4 \cdot 2 \cdot 2 \cdot 3 \cdot 2^2 \cdot 5 \cdot 2 \cdot 5$\\
$\phi(12!) = 2^{14} \cdot 3^5 \cdot 5^2$\\
$\phi(12!) = 99532800$


% Align example
% \begin{align*}
%     587 &= 1 \cdot 392 + 195 \\
%     392 &= 2 \cdot 195 + 2 \\
%     195 &= 97 \cdot 2 + 1 \\
%     97  &= 97 \cdot 1 + 0  \\
% \end{align*}

% Abschnittsweise definierte Funktionen
% \[ y = g(x) = 
%     \begin{cases} 
%     \frac{1}{2}x    & x \in ]-\infty; -2] \\
%     -2x+3           & x \in ]-2; 3]\\
%     5               & x \in ]3;\infty[
%  \end{cases}
% \]

% Matrix example
% \textbf{Korrektur:}\\
% $\mathbf{A} \odot \mathbf{B} =
% \begin{bmatrix}
%     1 & 1 & 1 \\
%     1 & 1 & 1 \\
%     1 & 0 & 1 \\
% \end{bmatrix}
% $

% tabular example 3 columns
% \renewcommand{\arraystretch}{1.5}
% \begin{center}
%     \begin{tabular}{ | m{12em} | m{12em} | m{12em} | }
%         \hline
%         1 & 2 & 3\\ 
%         \hline
%         1 & 2 & 3\\ 
%         \hline
%         1 & 2 & 3\\ 
%         \hline
%     \end{tabular}
% \end{center}


% tabular example 2 columns
% \renewcommand{\arraystretch}{1.5}
% \begin{center}
%     \begin{tabular}{ | m{17em} | m{17em} | }
%         \hline
%         1 & 2\\ 
%         \hline
%         1 & 2\\ 
%         \hline
%         1 & 2\\ 
%         \hline
%     \end{tabular}
% \end{center}

% \bibliography{}

\end{document}