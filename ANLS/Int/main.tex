\documentclass[12pt]{scrartcl}
\usepackage[ngerman]{babel}


\usepackage{amsmath, amssymb}

\usepackage{ulem} % double underline

\usepackage{array}  % for the tables

\usepackage{nameref}  % for referencing with name

\usepackage{hyperref}  % for hyperlinks

\usepackage{mathrsfs}

\usepackage{graphicx}  % for the images

\usepackage{glossaries}

\usepackage{xcolor, colortbl}

\usepackage{lmodern}
\usepackage{tcolorbox}

\definecolor{Gray}{gray}{0.85}

% hyperlinks
\hypersetup{
    colorlinks,
    citecolor=black,
    filecolor=black,
    linkcolor=black,
    urlcolor=black
}

\bibliographystyle{IEEetran}




\author{David Jäggli}

\title{Analysis Integralrechnung}



% ---------- Begin Main Document ----------- %



\begin{document}

\maketitle

\tableofcontents

\newpage
\section{Das unbestimmte Integral}
Bei der Integralrechnung haben wir die umgekehrte Aufgabenstellung als bei der Differenzialrechnung.
Anstatt Ableitung (quasi Aufleitung).\\
Fragenstellung: welche Funktion $F'(x)$ gibt abgeleitet $f(x)$.\\
\textbf{Beispiel:}\\
\[f(x) = x^3+2x^2+5x-6 \]
\[F(x) = \frac{1}{4}x^4+\frac{2}{3}x^3+\frac{5}{2}x^2-6x+c\] \\
\[ \int \sqrt[5]{x^4} \, dx = \int x^{\frac{4}{5}} = \frac{x^{\frac{4}{5} + 1}}{\frac{4}{5} + 1} = \frac{x^{\frac{9}{5}}}{\frac{9}{5}} = \uuline{\frac{9}{5} \cdot x^{\frac{5}{9}}}\]
\\
Wobei: $F(x) = \int f(x) \,dx$


Weil Konstante c fehlt ist es ein unbestimmtes Integral.

Nicht jede Funktion hat eine Stammfunktion.
\\


\renewcommand{\arraystretch}{1.5}
\begin{tcolorbox}
    \textbf{Man bezeichnet:}\\
    \begin{tabular}{cl}
        $f(x)$            & als \textbf{Integrand} = Funktion die hinter/unter dem Integral steht\\
        $\int f(x) \, dx$ & als \textbf{unbestimmtes Integral} \\
        $F(x) + c$        & als \textbf{Stammfunktion} \\
        $x$               & die \textbf{Integrationsvariable} \\
        $c$               & als \textbf{Integrationskonstante} \\
    \end{tabular}
\end{tcolorbox}

\subsection{Multiplikation von Funktionen}
Ein Produkt von Funktionen kann nicht einfach voneinander getrennt werden wie bei der Summe\\
Heisst:
\[\int f(x) \cdot g(x) \, dx \neq \int f(x) \, dx \cdot \int g(x) \, dx\]

\newpage

\subsection{Integration von weiteren Elementen}
Exponentielle Funktionen:
\begin{center}
    \renewcommand{\arraystretch}{2}
    \begin{tabular}{| m{5em} | m{25em} | }
        \hline
        $E_1$ & $\int e^x \, dx = e^x + c$ \\
        \hline
        $E_2$ & $\int e^{ax + b} \, dx = \frac{1}{a} \cdot e^{ax+b} + c$ \\
        \hline
        $E_3$ & $\int a^x \, dx = \frac{a^x}{ln(a)} + c$ \\
        \hline
    \end{tabular}
\end{center}
Logarithmische Funktionen:
\begin{center}
    \renewcommand{\arraystretch}{2}
    \begin{tabular}{| m{5em} | m{25em} | }
        \hline
        $L_1$ & $\int ln(x) \, dx = x \cdot ln(x) - x + c$ für $x \in \mathbb{R}_{+}^{*} $\\
        \hline
        $L_2$ & $\int ln(ax + b) \, dx = \frac{1}{a}[(ax + b) \cdot ln(ax+b) - (ax + b)] + c$ \\
        \hline
        $L_3$ & $\int log_ax \, dx = \frac{1}{ln(a)}(x \cdot ln(x) - x) + c$ für $x \in \mathbb{R}_{+}^{*}$ \\
        \hline
    \end{tabular}
\end{center}

\newpage

\section{Das bestimmte Integral}
\subsection{Die Berechnung des bestimmten Integrals}

% \renewcommand{\arraystretch}{1.5}
% \begin{center}
%     \begin{tabular}{ | m{12em} | m{12em} | m{12em} | }
%         \hline
%         1 & 2 & 3\\ 
%         \hline
%         1 & 2 & 3\\ 
%         \hline
%         1 & 2 & 3\\ 
%         \hline
%     \end{tabular}
% \end{center}


% \bibliography{quantum_ready}

\end{document}